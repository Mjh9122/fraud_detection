\documentclass{article}
\usepackage{graphicx}

\title{Comparing Graph Theoretic Features for Automated Credit Card Fraud Detection}
\author{Michael Holtz}
\date{\today}

\begin{document}

\maketitle

\section{Abstract}
In 2022, the Federal Trade Commission received over 440,000 reports of credit card fraud, an increase of over 50,000 over the previous year \cite{ftc2021,ftc2022}. The annual cost of credit card fraud in the United States has been estimated at XXXXXXX. These factors have lead researchers and institutions to develope novel techniques to detect credit card fraud. Using machine learning techniques to identify fraudulent transactions has been an area of research since at least 1994 and has been a hot topic as of late as the number of credit card transactions continues to rise year over year \cite{1994, Federal_Reserve_2023}. Specifically, Prusti, Das, and Rath found that the inclusion of three graph features increased the effectiveness of five supervised and two unsupervised machine learning algorithms \cite{graphdb}. In this paper we seek to evaluate many different graph features and feature selection algorithms to judge the best subset of features for each of the seven machine learning algorithms. 
\section{Introduction}
Trying to cite \cite{graphdb} \cite{Banksim}


\bibliographystyle{IEEEtran}
\bibliography{bibliography}

\end{document}